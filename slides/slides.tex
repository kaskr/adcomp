\documentclass[compress]{beamer}

\setlength{\unitlength}{\paperwidth}

\usepackage{graphicx, subfigure}
\usepackage{multimedia}
\usepackage{hyperref}
%\usepackage{verbatim}
\usepackage{listings}
\usepackage{caption} % For linebreaks in captions


\mode<presentation>
{
  \useoutertheme{split}
  \setbeamercolor{separation line}{use=structure,bg=structure.fg!50!bg}
  \usefonttheme{structurebold}
  
  \setbeamertemplate{navigation symbols}{}
  % or ...

  % \setbeamercovered{transparent}
  % or whatever (possibly just delete it)
}


%%\usepackage[danish]{babel}
%%\usepackage[utf-8]{inputenc}
\usepackage{times}
\usepackage[T1]{fontenc}


\title[RcppAD]
{A comparison between ADMD \& RcppAD}

\author[K. Kristensen, A. Nielsen, C.W. Berg ]% [Author, Another] % (optional, use only with lots of authors)
{Kasper Kristensen, Anders Nielsen, Casper W. Berg}

\date[September 2013] % (optional, should be abbreviation of conference name)
{September, 2013}

\begin{document}

\begin{frame}[plain]
  \titlepage
\end{frame}

\begin{frame}
\frametitle{RcppAD Intro Bullet List}

\begin{itemize}
  \item ADMB inspired R-package
  \item Implements Laplace approximation for random effects
  \item C++ Template based
  \item Combines extern libraries: cppAD, Eigen, CHOLMOD
  \item Automatic sparseness detection
  \item Parallelism through BLAS
  \item Parallel user templates
\end{itemize}


\end{frame}

\begin{frame}
  \frametitle{Linear regression code comparison example}
\end{frame}


\begin{frame}
  \frametitle{Example 2: formulae and plot}
\end{frame}

\begin{frame}
  \frametitle{Example 2: Code}
\end{frame}

\begin{frame}
  \frametitle{Example 2: Results (timings)}
\end{frame}

\begin{frame}
  \frametitle{Parallel user templates intro}
\end{frame}

\begin{frame}
  \frametitle{Parallel Code}
\end{frame}

\begin{frame}
  \frametitle{Results: benchmark plot}
\end{frame}


\begin{frame}
  \frametitle{Parallel Code with multicore package}
\end{frame}


\begin{frame}
  \begin{itemize}
    \item[+] Can handle very high dimensional problems (~$10^6$ random effects)
    \item[+] No \texttt{SEPARABLE\_FUNCTION} construct needed, fully automatic sparseness detection.
    \item[+] Full R integration -- no need for data+results import/export.
    \item[-] Standalone applications not possible.
    \item[+] Fast run times
    \item[-] Slow compile times.
    \item[+] Template based -- no code duplication for \texttt{df1b2variable}s etc.
    \item[-] Fewer built-in specialized functionalities (e.g. profile-likelihood, \texttt{sd\_report\_number} etc.)
    \item[+] Analytical Hessian for fixed effects.
    \item[+] High-level parallelization with \texttt{multicore} package. 
  \end{itemize}
\end{frame}


\end{document}